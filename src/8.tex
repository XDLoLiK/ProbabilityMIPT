\section{Характеристики случайной величины и случайного вектора}
\begin{definition}
	Распределением случайной величины (вектора) $\xi$ называется вероятностная мера $P_\xi$ на $(\mathbb{R},\,\mathcal{B}(\mathbb{R}))$ $(\mathbb{R}^n,\,\mathcal{B}(\mathbb{R}^n))$, определённая по правилу:
	\[P_\xi(B) = P(\xi \in B),\, B \in \mathcal{B}(\mathbb{R}) \; (\mathbb{R}^n)\]
\end{definition}

\begin{definition}
	Функцией распределения случайной величины $\xi$ называется
	\[F_\xi(x) = P(\xi \leq x) = P_\xi((-\infty,\, x])\]
\end{definition}

\begin{note}
	\[P(\xi_1 \leq x_1,\, \xi_2 \leq x_2) := P(\{\xi_1 \leq x_1\} \cap \{\xi_2 \leq x_2\})\]
\end{note}

\begin{definition}
	Если $\xi = (\xi_1,\,\cdots,\,\xi_n)$ -- случайный вектор, то его функцией распределения называется
	\[F_\xi(x_1,\,\cdots,\,x_n) = P_\xi((-\infty,\,x_1]\times\cdots\times (-\infty,\,x_n]) = P(\xi_1 \leq x_1,\, \cdots,\, \xi_n \leq x_n)\]
\end{definition}

\begin{definition}
	Случайная величина является
	\begin{itemize}
		\item Дискретной, если таково её распределение
		\item Абсолютно-непрерывной, если таково её распределение

		      В этом случае $\xi$ имеет плотность $p_\xi(t) \geq 0$:
		      \[F_\xi(x) = \int_{-\infty}^x p_\xi(t)dt\]
		\item Сингулярной, если таково её распределение
	\end{itemize}
\end{definition}

\begin{definition}
	Случайная величина $\xi$ называется простой, если она принимает конечное число значений. В этом случае $\xi$ имеет вид:
	\[\xi = \sum_{k = 1}^n x_k \mathbb{I}_{A_k}\]
	где $x_1,\,\cdots,\,x_n$ -- различные числа, $A_1,\,\cdots,\,A_n$ -- разбиение $\Omega$.
\end{definition}

\begin{definition}
	Пусть $\xi$ -- случайная величина (вектор) на $(\Omega,\, \mathcal{F},\,P)$. Сигма-алгеброй, порождённой $\xi$, называется
	\[\mathcal{F}_\xi = \{\xi^{-1}(B) :\: B \in \mathcal{B}(\mathbb{R})\} \: (\mathbb{R}^n)\]
	Это --- наименьшая сигма-алгебра на пространстве $\Omega$, относительно которой случайная величина $\xi$ всё ещё остаётся измеримой.
	Заметим, что $\mathcal{F}_\xi \subset \mathcal{F}$.
\end{definition}

\begin{definition}
	Случайная величина (вектор) $\eta$ является $\mathcal{F}_\xi$-измеримое, если $\mathcal{F}_\eta \subset \mathcal{F}_\xi$
\end{definition}

\begin{definition}
	Если $\xi$ -- это случайная величина, то положим
	\[\xi^+ := \max(\xi,\, 0);\;\;\; \xi^- := \max(-\xi,\, 0)\]
	Тогда, $\xi = \xi^+ - \xi^-$
\end{definition}

\begin{definition}
	Функция $\phi:\: \mathbb{R}^n \to \mathbb{R}^m$ называется борелевской, если прообраз любого борелевского множества есть борелевское множество:
	\[\forall B \in \mathcal{B}(\mathbb{R}^m) :\: \phi^{-1}(B) = \{x :\: \phi(x) \in B\} \in \mathcal{B}(\mathbb{R}^n)\]
\end{definition}

\begin{lemma} \label{BOREL_MEASURE}
	$\eta$ является $\mathcal{F}_\xi$-измеримой $\Leftrightarrow \exists$ борелевская функция $\phi$, такая что $\eta = \phi(\xi)$.
\end{lemma}

\begin{proof}
	$\Leftrightarrow$ Пусть $\eta = \phi(\xi)$ и $B \in \mathcal{B}(\mathbb{R})$. Тогда
	\[\{\eta \in B\} = \{\xi \in \stackrel{\in \mathcal{B}(\mathbb{R})}{\phi^{-1}(B)}\} \in \mathcal{F}_\xi \Rightarrow \mathcal{F}_\eta \subset \mathcal{F}_\xi\]
\end{proof}

\begin{theorem}
	О приближении простыми.

	\begin{enumerate}
		\item Пусть $\xi \geq 0$. Тогда $\exists$ последовательность $\mathcal{F}_\xi$-измеримых случайных величин $\{\xi_n,\, n \in \mathbb{N}\}$, такая что
		      \[0 \leq \xi_n \uparrow \xi\]
		\item Если $\xi$ -- произвольная случайная величина, то $\exists$ последовательность $\mathcal{F}_\xi$ измеримых простых случайных величин $\{\xi_n,\, n \in \mathbb{N}\}$, такая что
		      \[\forall n \in \mathbb{N}:\: |\xi_n| \leq |\xi|,\, \lim_{n \to +\infty}\xi_n = \xi\]
	\end{enumerate}
\end{theorem}

\begin{proof}
	\begin{enumerate}
		\item Предъявим $\xi_n$ в явном виде:
		      \[\xi_n = \sum_{k = 1}^{n2^n}\frac{k - 1}{2^n}\mathbb{I}_{\{\frac{k - 1}{2^n} \leq \xi < \frac{k}{2^n}\}}\]
		      Легко видеть, что $0 \leq \xi_n \leq \xi_{n + 1}$ и $\xi = \lim\limits_{n \to +\infty} \xi_n$. Кроме того, $\forall n:\: \xi_n$ -- борелевская функция от $\xi \Rightarrow \xi_n$ -- по (\ref{BOREL_MEASURE}) это $\mathcal{F}_\xi$-измеримая случайная величина.
		\item Приближаем $\xi^+$ и $\xi^-$ по предыдущему пункту, затем берём разность
	\end{enumerate}
\end{proof}
